\documentclass{beamer}

% Must be loaded first
\usepackage{tikz}

\usepackage[utf8]{inputenc}
\usepackage{textpos}

% Font configuration
\usepackage{fontspec}

\input{font.tex}

% Tikz for beautiful drawings
\usetikzlibrary{mindmap,backgrounds}
\usetikzlibrary{arrows.meta,arrows}
\usetikzlibrary{shapes.geometric}

% Minted configuration for source code highlighting
\usepackage{minted}
\setminted{highlightcolor=black!5, linenos}
\setminted{style=lovelace}

\usepackage[listings, minted]{tcolorbox}
\tcbset{left=6mm}

% Use the include theme
\usetheme{codecentric}

% Metadata
\title{Free All The Things}
\author{Markus Hauck}

% The presentation content
\begin{document}

\begin{frame}[noframenumbering,plain]
  \titlepage{}
\end{frame}

\section{Introduction}\label{sec:introduction}

\begin{frame}
\frametitle{Free All The Things}
\begin{itemize}
\item well known: free monads
\item maybe known: free applicatives
\item free monoids
\item free <you name it>
\end{itemize}
\end{frame}

\begin{frame}
  \frametitle{Goal Of This Talk}
  \begin{itemize}
  \item how many of you wrote a Free X
  \item how many of you used Free\ldots
    \begin{itemize}
    \item Monad
    \item Applicative
    \item Functor
    \item Boolean Algebra
    \item other?
    \end{itemize}
  \item Goal: explain the technique behind ``Free X''
  \item Be able to apply the ``pattern'' yourself
  \end{itemize}
\end{frame}

\begin{frame}
  \frametitle{The Road Ahead}
\end{frame}

\begin{frame}
  \frametitle{What Is Free}
  A free functor is left adjoint to a forgetful functor
  \vspace{1cm}
  \begin{center}
    What's the problem?
  \end{center}
\end{frame}

\begin{frame}[fragile]
  \frametitle{What Is Free} A free ``thing'' \textbf{FreeA} on a type
  \textit{A} is a \textit{A} and a function
  \begin{minted}[linenos=false]{scala}
    def inject(x: A): FreeA
  \end{minted}
  s. t. for any other ``thing'' \textit{B} and a function
  \begin{minted}[linenos=false]{scala}
    val f: A => B
  \end{minted}
  there exists a unique homomorphism \texttt{g} such that
  \begin{minted}[linenos=false]{scala}
    g.compose(inject) === f
  \end{minted}
\end{frame}

\begin{frame}[fragile]
  \frametitle{What Is Free}
  \begin{itemize}
  \item the good news: there is a recipe
    \begin{enumerate}
    \item create an AST for ops + vars
    \item modify it such that laws ensured during construction*
    \end{enumerate}
  \end{itemize}
\end{frame}

\begin{frame}
  \frametitle{Why Free}
  \begin{itemize}
  \item having a Free X is good for a number of reasons
  \item use Free X as if it was X
  \item but the program is reified into some (data-)structure
  \item this structure can often be analyzed and optimized
  \item many interpreters of the same program
  \end{itemize}
\end{frame}

\begin{frame}
  \frametitle{Scales of Power}
  \begin{itemize}
  \item the structures we will look at, are able to capture computations that have different power abilities
  \item monad: depend on previous values and branching
  \item applicative: fixed structure with arbitrary applicative effects in between
  \item functor: apply a function to the content
  \item monoid: limited power, but very flexible and composable
  \item surprise
  \end{itemize}
\end{frame}

\begin{frame}
  \frametitle{Disclaimer}
  \begin{itemize}
  \item we will mostly look at the data structure version of Free X
  \item the alternative is to use finally tagless representations (Next Talk)
  \end{itemize}
\end{frame}

\section{Freeing The Monad}\label{sec:free-monad}
\begin{frame}
  \begin{center}
    Freeing The Monad
  \end{center}
\end{frame}

\begin{frame}[fragile]
  \frametitle{Freeing The Monad}
  \begin{itemize}
  \item what are the operations?
  \end{itemize}
  \begin{center}
    \inputminted{scala}{snippets/monad-typeclass.scala}
  \end{center}
\end{frame}

\begin{frame}[fragile]
  \frametitle{Freeing The Monad}
\begin{itemize}
\item what are the laws?
\item (pseudocode)
\end{itemize}
    \begin{center}
\begin{minted}[autogobble]{scala}
// Left identity
pure(a).flatMap(f) === f(a)

// Right identity
fa.flatMap(pure) === fa

// Associativity
fa.flatMap(f).flatMap(g) ===
  fa.flatMap(a => f(a).flatMap(g))
\end{minted}
    \end{center}
\end{frame}

\begin{frame}[fragile]
  \frametitle{Freeing The Monad}
  \begin{itemize}
  \item todo: the minimal ``thing'' that has a \textit{Monad} instance
    \textbf{satisfies} the laws
  \item simple idea: capture as data
  \item any minimal combination works
  \end{itemize}
\end{frame}

\begin{frame}[fragile]
  \frametitle{Freeing The Monad}
  \begin{center}
    \inputminted{scala}{snippets/monad-typeclass.scala}
    \vspace{1cm}
    \inputminted{scala}{snippets/free-monad.scala}
  \end{center}
\end{frame}

\begin{frame}[fragile]
  \frametitle{Freeing The Monad}
  \begin{center}
    \inputminted{scala}{snippets/free-instance.scala}
  \end{center}
\end{frame}

\begin{frame}[fragile]
  \frametitle{Interpreter}
  \inputminted{scala}{snippets/free-interp.scala}
\end{frame}

\begin{frame}[fragile]
  \frametitle{What about the laws?}
  \begin{minted}{scala}
// The associativity law
fa.flatMap(f).flatMap(g) ===
  fa.flatMap(a => f(a).flatMap(g))
  \end{minted}
  \inputminted[autogobble]{scala}{snippets/what-about-laws.scala}
\end{frame}

\begin{frame}[fragile]
  \frametitle{What about the laws?}
  \begin{center}
    \includegraphics[width=0.6\textwidth]{static-images/scream.png}
  \end{center}
\end{frame}

\begin{frame}[fragile]
  \frametitle{The Laws}
  \begin{itemize}
  \item actually, we don't satisfy them
  \item programmer: after interpretation it's no longer visible
  \item mathematician: that's not the free monad!
  \item use them to make it faster
  \item tradeoff: during construction vs during interpretation
  \end{itemize}
\end{frame}

\begin{frame}[fragile]
  \frametitle{Faster Free Monads}
  \begin{itemize}
  \item common optimization: associate \texttt{flatMap}'s to the right
  \item avoids having to rebuild the tree repeatedly during construction
  \item how: during construction time
  \end{itemize}
\end{frame}

\begin{frame}[fragile]
  \frametitle{Faster Free Monads}
  \inputminted{scala}{snippets/opt-free-instance.scala}
\end{frame}

\begin{frame}
  \frametitle{Use Cases}
  \begin{itemize}
  \item DSL with monadic expressiveness
  \item branching, loops, basically everything
  \end{itemize}
\end{frame}

\begin{frame}
  \frametitle{Tradeoffs}
  \includegraphics[width=\textwidth]{ditaa/diagram-all.png}
  \begin{itemize}
  \item this is our base
  \item a lot of expressiveness in the DSL
  \item at the cost of the things you can do in the interpreter
  \end{itemize}
\end{frame}

\begin{frame}
  \frametitle{Freeing The Monad}
  \begin{itemize}
  \item that's it for the Monad
  \item what else?
  \end{itemize}
\end{frame}

\section{Freeing The Applicative}\label{sec:free-applicative}

\begin{frame}
  \frametitle{Freeing The Applicative}
  \begin{itemize}
  \item free monads are great, but also limited
  \item we can't analyze the programs
  \item how about a smaller gun?
  \end{itemize}
\end{frame}

\begin{frame}
  \frametitle{Freeing The Applicative}
  \begin{itemize}
  \item we follow the same pattern
  \item look at typeclass operations
  \item create datastructure
  \item ``interpreter''
  \end{itemize}
\end{frame}

\begin{frame}[fragile]
  \frametitle{The Applicative Class}
  \inputminted{scala}{snippets/applicative-typeclass.scala}
\end{frame}

\begin{frame}[fragile]
  \frametitle{Freeing The Applicative}
  \begin{itemize}
  \item again the same pattern: we model it as an ADT
  \end{itemize}
    \inputminted{scala}{snippets/free-applicative.scala}
  \begin{minted}[autogobble]{scala}
\end{minted}
\begin{itemize}
\item of course we also need the interpreter
\end{itemize}
\end{frame}

\begin{frame}
  \frametitle{Less Power?!}
  \begin{itemize}
  \item why would we consider Applicative if it's less powerful?
  \item less is more: we can inspect the AST
  \end{itemize}
\end{frame}

\begin{frame}[fragile]
  \frametitle{Freeing The Functor}
  \begin{itemize}
  \item we are well equipped by now
  \end{itemize}
\end{frame}

\begin{frame}[fragile]
  \frametitle{Freeing The Functor}
  \inputminted{scala}{snippets/free-functor.scala}
\end{frame}

\begin{frame}[fragile]
  \frametitle{Freeing The Functor}
  \begin{itemize}
  \item clean code alarm: only one subclass
  \item can we get rid of it?
  \end{itemize}
\end{frame}

\section{Freeing The Boolean Algebra}\label{sec:free-boolean-algebra}

\begin{frame}
  \frametitle{Disclaimer}
  \begin{itemize}
  \item Once upon a time: https://engineering.wingify.com/posts/Free-objects/
  \item really awesome article about free objects
  \item use free boolean algebra to define DSL for event predicates
  \item all credits to Chris Stucchio (@stucchio)
  \end{itemize}
\end{frame}

\begin{frame}
  \frametitle{Free Boolean Algebra}
  \begin{itemize}
  \item Wikipedia: boolean algebra + set of generators
  \item let's go
  \end{itemize}
\end{frame}

\begin{frame}
  \frame{Boolean Algebras}
  \begin{itemize}
  \item seen: common fp type classes
  \item apply our knowledge to another example: boolean algebras
  \end{itemize}
\end{frame}

\section{Conclusion}\label{sec:conclusion}

\begin{frame}
  \begin{center}
    \huge
    Your conclusion here
  \end{center}
\end{frame}

\end{document}

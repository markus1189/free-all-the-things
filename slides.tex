\documentclass{beamer}

% Must be loaded first
\usepackage{tikz}

\usepackage[utf8]{inputenc}
\usepackage{textpos}

% Font configuration
\usepackage{fontspec}

\input{font.tex}

% Tikz for beautiful drawings
\usetikzlibrary{mindmap,backgrounds}
\usetikzlibrary{arrows.meta,arrows}
\usetikzlibrary{shapes.geometric}

% Minted configuration for source code highlighting
\usepackage{minted}
\setminted{highlightcolor=black!5, linenos}
\setminted{style=lovelace}

\usepackage[listings, minted]{tcolorbox}
\tcbset{left=6mm}

% Use the include theme
\usetheme{codecentric}

% Metadata
\title{Free All The Things}
\author{Markus Hauck @markus1189}

\newcommand{\recipe}{%
  \begin{itemize}
  \item Define AST
  \item Add \texttt{Inject}
  \item Write Interpreter
  \item Check laws
  \end{itemize}
}

% The presentation content
\begin{document}

\begin{frame}[noframenumbering,plain]
  \titlepage{}
\end{frame}

\section{Introduction}\label{sec:introduction}

\begin{frame}
\frametitle{Free All The Things}
\begin{itemize}
\item well known: free monads
\item maybe known: free applicatives
\item free monoids
\item free <you name it>
\end{itemize}
\end{frame}

\begin{frame}
  \frametitle{This Talk}
  \begin{itemize}
  \item explain the technique behind ``Free X''
  \item apply the technique to different examples
  \item takeaway: it's easier than you thought
  \item Source Code: \url{https://github.com/markus1189/free-all-the-things/tree/lambdaconf}
  \end{itemize}
\end{frame}

\begin{frame}
  \frametitle{Let's Get Started}
  \begin{center}
    Ready?
  \end{center}
\end{frame}

\begin{frame}
  \frametitle{What's The Problem}
  \begin{quote}
    A free functor is left adjoint to a forgetful functor
  \end{quote}
  \only<2>{
  \begin{center}
    what's the problem?
  \end{center}
  \begin{center}
    \includegraphics[width=0.3\textwidth]{static-images/smirk.png}
  \end{center}
  }
\end{frame}

\begin{frame}[fragile]
  \frametitle{What Is Free} A free ``thing'' \textbf{FreeA} on a type(class)
  \textit{A} is a \textit{A} and a function
  \begin{minted}[linenos=false]{scala}
    def inject(x: A): FreeA
  \end{minted}
  such that for any other ``thing'' \textit{B} and a function
  \begin{minted}[linenos=false]{scala}
    val f: A => B
  \end{minted}
  there exists a unique homomorphism \texttt{g} such that
  \begin{minted}[linenos=false]{scala}
    g.compose(inject) === f
  \end{minted}
\end{frame}

\begin{frame}[fragile]
  \frametitle{What Is Free}
  \begin{itemize}
  \item still complicated!
  \item rejoice: there is a pretty mechanical recipe
    \recipe{}
  \end{itemize}
\end{frame}

\begin{frame}
  \frametitle{Why Free}
  \begin{itemize}
  \item nice API using typeclass
  \item use Free X as if it was X
  \item program reified into datastructure
  \item structure can be analyzed/optimized
  \item one program \textemdash{} many interpretations
  \end{itemize}
\end{frame}

\begin{frame}
  \frametitle{Disclaimer Before We Start}
  \begin{itemize}
  \item deep embeddings / initial encoding / data structure representation
  \item not: finally tagless, optimization
  \end{itemize}
\end{frame}

\section{Free Monad}\label{sec:free-monad}
\begin{frame}
  \begin{center}
    \Huge
    Freeing The Monad
  \end{center}
\end{frame}

\begin{frame}[fragile]
  \frametitle{The Monad Typeclass}
  \begin{center}
    \inputminted{scala}{snippets/monad-typeclass.scala}
  \end{center}
\end{frame}

\begin{frame}[fragile]
  \frametitle{Give Me The Laws}
    \begin{center}
\begin{minted}[autogobble]{scala}
// Left identity
pure(a).flatMap(f) === f(a)

// Right identity
fa.flatMap(pure) === fa

// Associativity
fa.flatMap(f).flatMap(g) ===
  fa.flatMap(a => f(a).flatMap(g))
\end{minted}
    \end{center}
\end{frame}

\begin{frame}[fragile]
  \frametitle{Applying The Recipe}
  \begin{center}
    \inputminted{scala}{snippets/monad-typeclass.scala}
  \end{center}
  \begin{itemize}
  \item remember our recipe
    \recipe{}
  \end{itemize}
\end{frame}

\begin{frame}[fragile]
  \frametitle{Freeing The Monad}
  \begin{center}
    \inputminted{scala}{snippets/free-monad.scala}
  \end{center}
\end{frame}

\begin{frame}[fragile]
  \frametitle{Freeing The Monad}
  \begin{center}
    \inputminted{scala}{snippets/free-instance.scala}
  \end{center}
\end{frame}

\begin{frame}[fragile]
  \frametitle{Interpreter}
  \inputminted{scala}{snippets/free-interp.scala}
\end{frame}

\begin{frame}[fragile]
  \frametitle{What about the laws?}
  \begin{minted}{scala}
// The associativity law
fa.flatMap(f).flatMap(g) ===
  fa.flatMap(fa, a => f(a).flatMap(g))
  \end{minted}
  \vfill{}
  \inputminted[autogobble]{scala}{snippets/what-about-laws.scala}
\end{frame}

\begin{frame}[fragile]
  \frametitle{What about the laws?}
  \begin{center}
    \includegraphics[width=0.3\textwidth]{static-images/scream.png}
  \end{center}
\end{frame}

\begin{frame}[fragile]
  \frametitle{The Laws}
  \begin{itemize}
  \item actually, we don't satisfy them
  \item programmer: after interpretation it's no longer visible
  \item mathematician: that's not the free monad!
  \item tradeoff: during construction vs during interpretation
  \end{itemize}
\end{frame}

\begin{frame}[fragile]
  \frametitle{The Right Free Monad}
  \begin{itemize}
  \item common transformation: associate \texttt{flatMap}'s to the right
  \item avoids having to rebuild the tree repeatedly during construction
  \item how: during construction time
  \end{itemize}
\end{frame}

\begin{frame}[fragile]
  \frametitle{Transforming Free Monads}
  \only<1>{\inputminted[highlightlines={3}, highlightcolor=yellow!40]{scala}{snippets/opt-free-instance.scala}}
  \only<2>{\inputminted[highlightlines={3, 6}, highlightcolor=yellow!40]{scala}{snippets/opt-free-instance.scala}}
\end{frame}

\begin{frame}
  \frametitle{We Freed Monads}
  \begin{itemize}
  \item DSL with monadic expressiveness
  \item context sensitive, branching, loops, fancy control flow
  \item familiarity with monadic style for DSL
  \item big drawback: interpreter has limited possibilities
  \end{itemize}
\end{frame}

\section{Free Applicative}\label{sec:free-applicative}
\begin{frame}
  \begin{center}
    \Huge
    Freeing The Applicative
  \end{center}
\end{frame}

\begin{frame}
  \frametitle{Freeing The Applicative}
  \begin{itemize}
  \item free monads are great, but also limited
  \item we can't analyze the programs
  \item how about a smaller abstraction?
  \end{itemize}
\end{frame}

\begin{frame}
  \frametitle{Recall}
  \recipe{}
\end{frame}

\begin{frame}[fragile]
  \frametitle{The Applicative Typeclass}
  \inputminted{scala}{snippets/applicative-typeclass.scala}
\end{frame}

\begin{frame}[fragile]
  \frametitle{AST for FreeApplicative}
    \inputminted{scala}{snippets/free-applicative.scala}
  \begin{minted}[autogobble]{scala}
\end{minted}
\end{frame}

\begin{frame}[fragile]
  \frametitle{Laws}
\begin{minted}[autogobble]{scala}
// identity
Ap(Pure(identity), v) === v

// composition
Ap(Ap(Ap(Pure(_.compose), u), v), w) ===
  Ap(u, Ap(v, w))

// homomorphism
Ap(Pure(f), Pure(x)) === Pure(f(x))

// interchange
Ap(u, Pure(y)) === Ap(Pure(_(y)), u)
\end{minted}
\end{frame}

\begin{frame}
  \frametitle{Don't Forget The Laws}
  \only<1>{\inputminted[highlightlines={5, 7}, highlightcolor=yellow!40]{scala}{snippets/opt-free-ap-instance.scala}}
\end{frame}

\begin{frame}
  \frametitle{Running FreeApplicatives}
  \inputminted{scala}{snippets/freeap-interp.scala}
\end{frame}

\begin{frame}
  \frametitle{We Freed Applicatives}
  \begin{itemize}
  \item DSL with applicative expressiveness
  \item context insensitive
  \item pure computation over effectful arguments
  \item more freedom during interpretation
  \end{itemize}
\end{frame}

\section{Free Functor}\label{sec:free-functor}
\begin{frame}
  \begin{center}
    \Huge
    Freeing The Functor
  \end{center}
\end{frame}

\begin{frame}[fragile]
  \frametitle{And Once Again}
  \recipe{}
\end{frame}

\begin{frame}[fragile]
  \frametitle{The Functor Typeclass}
  \inputminted{scala}{snippets/functor-typeclass.scala}
\end{frame}

\begin{frame}[fragile]
  \frametitle{Freeing The Functor}
  \only<1>{\inputminted[highlightlines={6-7}, highlightcolor=yellow!40]{scala}{snippets/free-functor1.scala}}
\end{frame}

\begin{frame}[fragile]
  \frametitle{Freeing The Functor}
  \inputminted{scala}{snippets/free-functor2.scala}
\end{frame}

\begin{frame}[fragile]
  \frametitle{Clean Code Police}
  \begin{center}
    \includegraphics[width=0.3\textwidth]{static-images/police.png}
  \end{center}
  \begin{itemize}
  \item only one subclass?
  \end{itemize}
\end{frame}

\begin{frame}[fragile]
  \frametitle{Freeing The Functor}
  \inputminted{scala}{snippets/free-functor3.scala}
\end{frame}

\begin{frame}[fragile]
  \frametitle{Freeing The Functor}
  \inputminted{scala}{snippets/free-functor4.scala}
\end{frame}

\begin{frame}[fragile]
  \frametitle{Free Functor Instance}
  \only<1>{\inputminted[highlightlines={6-8}, highlightcolor=yellow!40]{scala}{snippets/functor-coyoneda.scala}}
\end{frame}

\begin{frame}[fragile]
  \frametitle{Free Functor Interpreter}
  \only<1>{\inputminted{scala}{snippets/coyoneda-interp.scala}}
\end{frame}

\begin{frame}
  \frametitle{We Freed Functors}
  \begin{itemize}
  \item DSL with hmm functorial expressiveness?
  \item map fusion! (functor law)
  \item boring interpreter, though
  \item still fun!
  \end{itemize}
\end{frame}

\section{Free Monoid}
\label{sec:free-monoid}

\begin{frame}
  \begin{center}
    \Huge
    Freeing The Monoid
  \end{center}
\end{frame}

\begin{frame}
  \frametitle{The Monoid Typeclass}
  \inputminted{scala}{snippets/monoid-typeclass.scala}
\end{frame}

\begin{frame}
  \frametitle{The Free Monoid \textemdash{} First Try}
  \inputminted{scala}{snippets/free-monoid-1.scala}
\end{frame}

\begin{frame}[fragile]
  \frametitle{The Laws}
  \begin{minted}{scala}
// left identity
empty |+| x === x

// right identity
x |+| empty === x

// associativity
1 |+| (2 |+| 3) === (1 |+| 2) |+| 3
  \end{minted}
\end{frame}

\begin{frame}
  \frametitle{The Laws and Free Monoid}
  \begin{itemize}
  \item let's try to enforce those laws in our structure
  \item goal: correct by construction
  \item arbitrary decision: associate left vs \textbf{right}
  \end{itemize}
\end{frame}

\begin{frame}
  \frametitle{Fixing Associativity}
  \inputminted[highlightlines={11,13,7}, highlightcolor=yellow!40]{scala}{snippets/free-monoid-2.scala}
\end{frame}

\begin{frame}
  \frametitle{The Problem With Neutral Elements}
  \begin{itemize}
  \item get rid completely? not possible
  \item limit ourselves to a single element
  \item restrict \texttt{Combine} to have only real values on the left side
  \item goal: minimal canonical structure
  \end{itemize}
\end{frame}

\begin{frame}
  \frametitle{Minimizing Structure \textemdash{} Extract Inject}
  \inputminted[highlightlines={1, 7}, highlightcolor=yellow!40]{scala}{snippets/free-monoid-3.scala}
\end{frame}

\begin{frame}
  \frametitle{Minimizing Structure \textemdash{} Remove Inject}
  \inputminted[highlightlines={5}, highlightcolor=yellow!40]{scala}{snippets/free-monoid-4.scala}
\end{frame}

\begin{frame}
  \frametitle{The Monoid Instance}
  \inputminted{scala}{snippets/free-monoid-instance.scala}
\end{frame}

\begin{frame}
  \frametitle{Minimizing Structure \textemdash{} List}
  \inputminted{scala}{snippets/free-monoid-5.scala}
\end{frame}

\begin{frame}
  \frametitle{We Freed Monoids}
  \begin{itemize}
  \item DSL for combining things
  \item works beautifully with folds
  \item interpretation can be in parallel (associativity)
  \end{itemize}
\end{frame}

\section{Free Boolean Algebra}\label{sec:free-boolean-algebra}

\begin{frame}
  \frametitle{Now That We Can Free Anything}
  \begin{center}
    \includegraphics[width=0.3\textwidth]{static-images/thinking.png}
    \vfill
    {\Huge What should we free?}
  \end{center}
\end{frame}

\begin{frame}
  \frametitle{Credit Where It's Due}
  \begin{itemize}
  \item Once upon a time:
    https://engineering.wingify.com/posts/Free-objects/
  \item use \textbf{free boolean algebra} to define DSL for event
    predicates
  \item credits to Chris Stucchio (@stucchio)
  \end{itemize}
\end{frame}

\begin{frame}
  \frametitle{Let's Free A Boolean Algebra}
  \recipe{}
\end{frame}

\begin{frame}
  \frametitle{Boolean Algebras}
  \inputminted{scala}{snippets/boolean-algebra.scala}
\end{frame}

\begin{frame}
  \frametitle{Free Boolean Algebra}
  \inputminted{scala}{snippets/free-bool.scala}
\end{frame}

\begin{frame}
  \frametitle{Free Boolean Algebra}
  \only<1>{\inputminted[highlightlines={1-2}, highlightcolor=yellow!40]{scala}{snippets/free-bool-interp.scala}}
  \only<2>{\inputminted[highlightlines={4,5}, highlightcolor=yellow!40]{scala}{snippets/free-bool-interp.scala}}
  \only<3>{\inputminted[highlightlines={6}, highlightcolor=yellow!40]{scala}{snippets/free-bool-interp.scala}}
  \only<4>{\inputminted[highlightlines={7}, highlightcolor=yellow!40]{scala}{snippets/free-bool-interp.scala}}
  \only<5>{\inputminted[highlightlines={9,11}, highlightcolor=yellow!40]{scala}{snippets/free-bool-interp.scala}}
\end{frame}

\begin{frame}
  \frametitle{Using Free Bool}
  \begin{itemize}
  \item that was simple (though boilerplate-y)
  \item what can we do with our new discovered structure
  \item reminder: boolean operators
    \begin{itemize}
    \item true, false
    \item and, or
    \item xor, implies, nand, nor, nxor
    \end{itemize}
  \end{itemize}
\end{frame}

\begin{frame}
  \frametitle{Free Bool Example: Search}
  \inputminted{scala}{snippets/search-predicate.scala}
\end{frame}

\begin{frame}
  \frametitle{Free Bool Example: Search}
  \begin{itemize}
  \item after sneaking in syntactic sugar behind the scenes:
  \end{itemize}
  \inputminted{scala}{snippets/example-search-predicate.scala}
\end{frame}

\begin{frame}
  \frametitle{The Site Type}
  \inputminted{scala}{snippets/site-type.scala}
  \begin{itemize}
  \item all predicates are run against a collection of these
  \end{itemize}
\end{frame}

\begin{frame}
  \frametitle{Free Bool Example: Search}
  \only<1>{\inputminted[highlightlines={4}, highlightcolor=yellow!40]{scala}{snippets/eval-search-predicate.scala}}
  \only<2>{\inputminted[highlightlines={5}, highlightcolor=yellow!40]{scala}{snippets/eval-search-predicate.scala}}
  \only<3>{\inputminted[highlightlines={6}, highlightcolor=yellow!40]{scala}{snippets/eval-search-predicate.scala}}
  \only<4>{\inputminted[highlightlines={7}, highlightcolor=yellow!40]{scala}{snippets/eval-search-predicate.scala}}
\end{frame}

\begin{frame}
  \frametitle{But Wait There's More}
  \begin{itemize}
  \item short circuiting and other optimization
  \item what if you don't have all the information?
    \begin{itemize}
    \item partially evaluate predicates
    \item if evaluates successfully, done
    \item else, send it on
    \end{itemize}
  \item core language vs extension
    \begin{itemize}
    \item Chris also demonstrates extension
    \item translate a rich language to base instructions
    \item with all the advantages
    \end{itemize}
  \end{itemize}
\end{frame}

\begin{frame}
  \frametitle{Optimizing Boolean Algebras}
  \inputminted{scala}{snippets/optimizing-boolean-algebras.scala}
\end{frame}

\begin{frame}
  \frametitle{Partial Evaluation}
  \begin{itemize}
  \item idea: you might have only partial information
  \item evaluate as much as possible
  \item optimal: we can already reduce without needing more information
  \item otherwise: send it on (JSON, Protobuf, \ldots{})
  \end{itemize}
\end{frame}

\begin{frame}
  \frametitle{Partial Evaluation}
  \inputminted{scala}{snippets/partial-evaluator.scala}
\end{frame}

\begin{frame}
  \frametitle{Partial Evaluation}
  \inputminted{scala}{snippets/partially.scala}
\end{frame}

\begin{frame}
  \frametitle{We Freed Boolean Algebras}
  \begin{itemize}
  \item good example of underused free structure
  \item partial evaluation
  \item serialize the AST (JSON, Protobuf, Avro, \ldots{})
  \item exercise: minimize AST representation
  \end{itemize}
\end{frame}

\section{Conclusion}\label{sec:conclusion}

\begin{frame}
  \frametitle{Resources}
  \begin{itemize}
  \item Free Boolean Algebra by Chris Stucchio \url{https://engineering.wingify.com/posts/Free-objects/}
  \item Source Code: \url{https://github.com/markus1189/free-all-the-things/tree/lambdaconf}
  \end{itemize}
\end{frame}

\begin{frame}
  \begin{center}
    \Huge
    Go And Free All The Things!
  \end{center}
  \begin{center}
    Markus Hauck (@markus1189)
  \end{center}
\end{frame}

\begin{frame}
  \tableofcontents{}
\end{frame}

\appendix{}

\section*{Bonus}\label{sec:bonus}

\begin{frame}
  \frametitle{Going Deeper}
  \begin{itemize}
  \item try to encode one of the normal forms for boolean algebras
  \item try to remove \texttt{Inject} cases from Monad and Applicative
  \item free Magmas
  \item define free X using alternative minimal set of ops of the typeclass
  \end{itemize}
\end{frame}

\end{document}

\documentclass{beamer}

% Must be loaded first
\usepackage{tikz}

\usepackage[utf8]{inputenc}
\usepackage{textpos}

% Font configuration
\usepackage{fontspec}

\input{font.tex}

% Tikz for beautiful drawings
\usetikzlibrary{mindmap,backgrounds}
\usetikzlibrary{arrows.meta,arrows}
\usetikzlibrary{shapes.geometric}

% Minted configuration for source code highlighting
\usepackage{minted}
\setminted{highlightcolor=black!5, linenos}
\setminted{style=lovelace}

\usepackage[listings, minted]{tcolorbox}
\tcbset{left=6mm}

% Use the include theme
\usetheme{codecentric}

% Metadata
\title{Free All The Things}
\author{Markus Hauck}

\newcommand{\recipe}{%
  \begin{itemize}
  \item create an AST for ops + vars
  \item provide a function to ``inject'' things
  \item define an interpreter that eliminates the AST (homomorphism)
  \item look at the laws
  \end{itemize}
}

% The presentation content
\begin{document}

\begin{frame}[noframenumbering,plain]
  \titlepage{}
\end{frame}

\section{Introduction}\label{sec:introduction}

\begin{frame}
\frametitle{Free All The Things}
\begin{itemize}
\item well known: free monads
\item maybe known: free applicatives
\item free monoids
\item free <you name it>
\end{itemize}
\end{frame}

\begin{frame}
  \frametitle{Goal Of This Talk}
  \begin{itemize}
  \item how many of you wrote a Free X
  \item how many of you used Free\ldots
    \begin{itemize}
    \item Monad
    \item Applicative
    \item Functor
    \item other?
    \end{itemize}
  \item Goal: explain the technique behind ``Free X''
  \end{itemize}
\end{frame}

\begin{frame}
  \frametitle{The Road Ahead}
\end{frame}

\begin{frame}
  \frametitle{What's The Problem}
  \begin{center}
    A free functor is left adjoint to a forgetful functor
  \end{center}
  \begin{center}
    what's the problem?
  \end{center}
  \begin{center}
    \includegraphics[width=0.3\textwidth]{static-images/smirk.png}
  \end{center}
\end{frame}

\begin{frame}[fragile]
  \frametitle{What Is Free} A free ``thing'' \textbf{FreeA} on a type
  \textit{A} is a \textit{A} and a function
  \begin{minted}[linenos=false]{scala}
    def inject(x: A): FreeA
  \end{minted}
  such that for any other ``thing'' \textit{B} and a function
  \begin{minted}[linenos=false]{scala}
    val f: A => B
  \end{minted}
  there exists a unique homomorphism \texttt{g} such that
  \begin{minted}[linenos=false]{scala}
    g.compose(inject) === f
  \end{minted}
\end{frame}

\begin{frame}[fragile]
  \frametitle{What Is Free}
  \begin{itemize}
  \item still sounds complicated?
  \item there is a recipe
    \recipe{}
  \end{itemize}
\end{frame}

\begin{frame}
  \frametitle{Why Free}
  \begin{itemize}
  \item use Free X as if it was X
  \item program reified into (data-)structure
  \item structure can be analyzed/optimized
  \item one program \textemdash{} many interpretations
  \end{itemize}
\end{frame}

\begin{frame}
  \frametitle{Disclaimer Before We Start}
  \begin{itemize}
  \item this talk: deep embeddings / initial encoding / data structure representation
  \item alternative: finally tagless
  \item not this talk: optimization of free structures
  \end{itemize}
\end{frame}

\section{Freeing The Monad}\label{sec:free-monad}
\begin{frame}
  \begin{center}
    \Huge
    Freeing The Monad
  \end{center}
\end{frame}

\begin{frame}[fragile]
  \frametitle{Monad Operations}
  \begin{center}
    \inputminted{scala}{snippets/monad-typeclass.scala}
  \end{center}
\end{frame}

\begin{frame}[fragile]
  \frametitle{Give Me The Laws}
    \begin{center}
\begin{minted}[autogobble]{scala}
// Left identity
pure(a).flatMap(f) === f(a)

// Right identity
fa.flatMap(pure) === fa

// Associativity
fa.flatMap(f).flatMap(g) ===
  fa.flatMap(a => f(a).flatMap(g))
\end{minted}
    \end{center}
\end{frame}

\begin{frame}[fragile]
  \frametitle{Applying The Recipe}
  \begin{center}
    \inputminted{scala}{snippets/monad-typeclass.scala}
  \end{center}
  \begin{itemize}
  \item now comes our recipe
    \recipe{}
  \end{itemize}
\end{frame}

\begin{frame}[fragile]
  \frametitle{Freeing The Monad}
  \begin{center}
    \inputminted{scala}{snippets/free-monad.scala}
  \end{center}
\end{frame}

\begin{frame}[fragile]
  \frametitle{Freeing The Monad}
  \begin{center}
    \inputminted{scala}{snippets/free-instance.scala}
  \end{center}
\end{frame}

\begin{frame}[fragile]
  \frametitle{Interpreter}
  \inputminted{scala}{snippets/free-interp.scala}
\end{frame}

\begin{frame}[fragile]
  \frametitle{What about the laws?}
  \begin{minted}{scala}
// The associativity law
FlatMap(FlatMap(fa, f), g) ===
  FlatMap(fa, a => FlatMap(f(a), g))
  \end{minted}
  \vfill{}
  \inputminted[autogobble]{scala}{snippets/what-about-laws.scala}
\end{frame}

\begin{frame}[fragile]
  \frametitle{What about the laws?}
  \begin{center}
    \includegraphics[width=0.3\textwidth]{static-images/scream.png}
  \end{center}
\end{frame}

\begin{frame}[fragile]
  \frametitle{The Laws}
  \begin{itemize}
  \item actually, we don't satisfy them
  \item programmer: after interpretation it's no longer visible
  \item mathematician: that's not the free monad!
  \item tradeoff: during construction vs during interpretation
  \end{itemize}
\end{frame}

\begin{frame}[fragile]
  \frametitle{The Right Free Monad}
  \begin{itemize}
  \item common transformation: associate \texttt{flatMap}'s to the right
  \item avoids having to rebuild the tree repeatedly during construction
  \item how: during construction time
  \end{itemize}
\end{frame}

\begin{frame}[fragile]
  \frametitle{Transforming Free Monads}
  \inputminted[highlightlines={5-6}, highlightcolor=yellow]{scala}{snippets/opt-free-instance.scala}
\end{frame}

\begin{frame}
  \frametitle{Use Cases}
  \begin{itemize}
  \item DSL with monadic expressiveness
  \item context sensitive, branching, loops, fancy control flow
  \item familiarity with monadic style for DSL
  \item big drawback: interpreter has limited possibilities
  \end{itemize}
\end{frame}

\section{Freeing The Applicative}\label{sec:free-applicative}
\begin{frame}
  \begin{center}
    \Huge
    Freeing The Applicative
  \end{center}
\end{frame}

\begin{frame}
  \frametitle{Freeing The Applicative}
  \begin{itemize}
  \item free monads are great, but also limited
  \item we can't analyze the programs
  \item how about a smaller abstraction?
  \end{itemize}
\end{frame}

\begin{frame}
  \frametitle{Recall}
  \begin{itemize}
  \item we follow the same pattern
  \end{itemize}
  \recipe{}
\end{frame}

\begin{frame}[fragile]
  \frametitle{The Applicative Class}
  \inputminted{scala}{snippets/applicative-typeclass.scala}
\end{frame}

\begin{frame}[fragile]
  \frametitle{AST for FreeApplicative}
    \inputminted{scala}{snippets/free-applicative.scala}
  \begin{minted}[autogobble]{scala}
\end{minted}
\end{frame}

\begin{frame}[fragile]
  \frametitle{Laws}
\begin{minted}[autogobble]{scala}
// identity
Ap(Pure(identity), v) === v

// composition
Ap(Ap(Ap(Pure(_.compose), u), v), w) ===
  Ap(u, Ap(v, w))

// homomorphism
Ap(Pure(f), Pure(x)) === Pure(f(x))

// interchange
Ap(u, Pure(y)) === Ap(Pure(_(y)), u)
\end{minted}
\end{frame}

\begin{frame}
  \frametitle{Less Power?!}
  \begin{itemize}
  \item why would we consider Applicative if it's less powerful?
  \item less is more: we can inspect the AST
  \end{itemize}
\end{frame}

\begin{frame}[fragile]
  \frametitle{Freeing The Functor}
  \begin{itemize}
  \item we are well equipped by now
  \end{itemize}
\end{frame}

\begin{frame}[fragile]
  \frametitle{Freeing The Functor}
  \inputminted{scala}{snippets/free-functor.scala}
\end{frame}

\begin{frame}[fragile]
  \frametitle{Freeing The Functor}
  \begin{itemize}
  \item clean code alarm: only one subclass
  \item can we get rid of it?
  \end{itemize}
\end{frame}

\section{Freeing The Boolean Algebra}\label{sec:free-boolean-algebra}

\begin{frame}
  \frametitle{Disclaimer}
  \begin{itemize}
  \item Once upon a time: https://engineering.wingify.com/posts/Free-objects/
  \item really awesome article about free objects
  \item use free boolean algebra to define DSL for event predicates
  \item all credits to Chris Stucchio (@stucchio)
  \end{itemize}
\end{frame}

\begin{frame}
  \frametitle{Free Boolean Algebra}
  \begin{itemize}
  \item Wikipedia: boolean algebra + set of generators
  \item let's go
  \end{itemize}
\end{frame}

\begin{frame}
  \frame{Boolean Algebras}
  \begin{itemize}
  \item seen: common fp type classes
  \item apply our knowledge to another example: boolean algebras
  \end{itemize}
\end{frame}

\section{Conclusion}\label{sec:conclusion}

\begin{frame}
  \frametitle{Tradeoffs}
  \includegraphics[width=\textwidth]{ditaa/diagram-all.png}
  \begin{itemize}
  \item this is our base
  \item a lot of expressiveness in the DSL
  \item at the cost of the things you can do in the interpreter
  \end{itemize}
\end{frame}

\begin{frame}
  \begin{center}
    \huge
    Your conclusion here
  \end{center}
\end{frame}

\end{document}
